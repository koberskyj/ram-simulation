\chapter{Použité technologie}
\label{sec:technologies}
V této kapitole jsou popsány základní technologie a jejich balíčky, které jsou v této práci použity.

\section{JSON}
Jedná se o standardní formát textu, přes který lze reprezentovat strukturovaná data. 
Výhodou oproti binárnímu kódu či standardnímu textového souboru je čitelnost, 
kdy jsou definovaná jasná pravidla formátu, který je snadno čitelný jak pro uživatele, tak počítače.
Tento formát je založen na základně syntaxe JavaScript (JS) objektů a slouží především pro ukládání dat a komunikaci mezi zařízeními \cite{bray_json}.

\section{TypeScript}
TypeScript je o open-source jazyk vyvinutý firmou Microsoft, který především rozšiřuje JS syntaxi o typy a jejich kontrolu.
Zdrojový kód se nejprve přeloží (transpiluje) do JS kódu, který lze následne spustit ve webovém prohlížeči (nebo na serveru přes například Node.js). 
Mezi hlavní přednosti tohoto jazyka patří zejména dřívější odhalení chyb díky typové kontrole a našeptávání během psaní kódu \cite{w3schools_typescript}.

\section{Webové technologie}
Webová stránka v dnešní době již není většinou skládaná pouze z HTML a CSS, 
mnohé stránky se neobejdou bez různé interaktivity pomocí jazyka JavaScript, případně WebAssembly.
Stránka pak může díky této interaktivity komunitovat přímo s lokálním počítačem či internetem, 
může provádět různé dynamické upravy v DOM (Document Object Model) a slouží i ke zprostředkování složitějších animací.

\subsection{Použité API a knihovny}
Pro zjednodušení práce a přidání možnosti ukládání dat byly použity tyto webové technologie:
\begin{itemize}
  \item \textbf{localStorage}, sloužící pro lokální dlouhodobé uložení dat ve formátu klíč-hodnota, která je formátu JSON \cite{mozilla_localstorage}.
  \item \textbf{Shadcn}, pro již předpřipravené komponenty, jako jsou například tlačítka, ikonky a vyskakovací okna. Samotná knihovna využívá Radix UI a Lucide ikonky \cite{shadcn_introduction}.
  \item \textbf{Tailwind CSS}, který slouží pro jednoduché stylování přímo přes třídy na jednotlivých HTML elementech nebo jejich ekvivalentních prvků \cite{tailwindcss}.
\end{itemize}

\subsection{React.js}
Samotná aplikace je napsána za pomocí frameworku React.js, který výrazně zjednodušuje zejména interaktivitu aplikace a rozděluje kód do jednotivých komponent. 

React.js slouží ke tvorbě webových uživatelských rozhraní v JS nebo, v tomto případě, v jazyce TypeScript.
Hlavní myšlenkou knihovny je deklarativní a komponentový přístup - rozhraní se skládá z malých, znovupoužitelných kompopent, 
které popisují, jak má aplikace vypadat pro konkrétní stav dat. 
HTML elementy komponent a i celkově jednotlivé kompomenty jsou v tomto případě reprezentovány ekvivalentní strukturou JSX/TSX s minimálními změnami.
Díky této myšlence lze snadno vytvářet komplexní UI prvky, které jsou na pozadí složeny z menších a menších prvků.
React kód je převážně spouštěn na straně uživatele v prohlížeči, 
avšak existují i alternativy v podobě mobilních aplikací přes React Native nebo kódů pro server přes Next.js.
Při změně jednotlivých komponent React efektivně aktualizuje pouze dotčené části DOM (Document Object Model) 
pomocí své vlastní implementace zvané \enquote{virtual DOM} \cite{mozilla_react}.

\section{Git}
Jedná se o jeden z nejrozšířenějších verzovacích systémů, který výrazně usnadňuje práci v týmu, řešení záloh evidenci jednotlivých změn.
Data mohou přes něj mohou být uložena jak lokálně, tak i na serveru \cite{git}.
V této práci byl použit server \textbf{GitHub}, který umožňuje správu privátních i věřejných projektů, 
obsahuje uživatelky přívětivé rozhraní a má i několik dodatečných funkcí navíc, jako jsou například GitHub pages \cite{github}.

V této práci je použit verzovací systém zejména kvůli tomu, aby měl vedoucí práce vždy k dispozici aktuální verzi projektu, včetně historie aktivity. 
Aplikace je také zveřejněna na již zmíněném GitHub Pages, jelikož je tato funkce zdarma a odpadá tak nutnost řešit hosting na jiných službách.

\subsection{GitHub Pages}
GitHub Pages slouží pro automatické publikování webových stránek s předpřipravenými či vlastními URL \cite{githubpages}. 
Vzhledem k použití frameworku je však nutné aplikaci nejdříve sestavit, a pak až následně publikovat. 
Aplikace se publikuje do vlastní větve v repozitáři projektu, takže nemusí být součástí zdrojových kódů.
