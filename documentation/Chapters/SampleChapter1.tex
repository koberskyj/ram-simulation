\chapter{Turingovy stroje a stroje RAM}
\label{sec:theory}

\section{Turingův stroj}
Jedná se o idealizovaný model počítače, který lze použít ke zkoumání hranic algoritmicky řešitelných úloh. 
Stroj popsal v roce 1936 Alan Turing a od té doby je to jeden z klíčových formálních nástrojů pro
 definici pojmu \texttt{algoritmus} a pro charakterizaci rekurzivně vyčíslitelných jazyků \cite{geeksforgeeks_turing}.
Je na něm postavena \textbf{Churchova-Turingova teze}, podle které může být každý algoritmus realizován Turingovým strojem. 
Programovací jazyky a stroje, které umožňují vyjádřit libovolný takovýto algoritmus, se označují jako \textbf{Turingovsky úplné} \cite{sawa_teoreticka}.

Existuje několik různých variant Turingova stroje, všechny ale obsahují nějakou verzi \textit{nekonečné pásky}, \textit{čtecí hlavu} a \textit{přepisovací pravidla}.
V této práci je Turingův stroj implementován s oboustranně nekonečnou páskou, samotná simulace je však limitována na stroj s jednostranně nekonečnou páskou. 
Více je konkrétní implementace popsána v kapitole \ref{sec:machine_impl}.

\subsection{Definice stroje}
Formálně je Turingův stroj definován jako šestice $M = (Q, \Sigma, \Gamma, \delta, q_0, F)$, kde \cite{sawa_teoreticka}: 
\begin{itemize}
	\item $Q$ je konečná neprázdná množina stavů,
	\item $\Gamma$ je konečná neprázdná množina páskových symbolů,
	\item $\Sigma \subseteq \Gamma$ je konečná neprázdná množina vstupních symbolů,
	\item $\delta : (Q - F) \times \Gamma \rightarrow Q \times \Gamma \times \{-1, 0, +1\}$ je přechodová funkce,
	\item $q_0 \in Q$ je počáteční stav,
	\item $F \subseteq Q$ je množina koncových stavů.
\end{itemize}
Výsledek rozdílu $\Gamma - \Sigma$ je vždy speciální znak $\square$, který označuje prázdný znak (blank).

\subsection{Příklady Turingových strojů}
% Lze pak doplnit o výpočet a konfiguraci
Aplikace obsahuje základních 5 příkladů, které pracují následovně:
\begin{itemize}
	\item \textbf{Shodné délky}\footnote{Jedná se o ukázkový příklad z prezentace předmětu teoretické informatiky} - stroj přijímá slova, kde se symboly $a, b, c$ opakují n-krát za sebou,
	\item \textbf{Zrcadlit} - stroj zrcadlí symboly $a, b$ směrem doprava,
	\item \textbf{Kopírovat} - stroj kopíruje symboly $a, b$ směrem doprava,
	\item \textbf{Palindrom} - stroj přijímá slova složená ze symbolů $a, b$, která jsou z obou stran stejná,
	\item \textbf{Sudý počet a} - stroj přijímá slova, kde se vyskytuje sudý počet symbolu $a$.
\end{itemize}

\section{RAM stroj}
RAM stroj (Random-Access Machine) představuje turingovsky úplný, idealizovaný model počítače. 
Skládá se z \emph{programové jednotky} (sekvence instrukcí), \emph{pracovní paměti} a \emph{vstupní} a \emph{výstupní} pásky.
Buňky paměti i pásky obsahují pouze celá čísla ($\mathbb{N}$), nelze do nich tedy uložit znak. 
Pracovní paměť slouží pro vstup i výstup a je indexována od $0$ $(R_0)$ do $n$ $(R_n)$. Vstupní páska slouží pouze pro čtení a naopak výstupní pouze pro zápis.
Stejně jako u Turingova stroje existuje i zde řada modifikovaných definic RAM stroje.
Stroj navíc obsahuje ukazatel na právě prováděnou instrukci v programové jednotce (IP) a v základu obsahuje tyto instrukce \cite{sawa_teoreticka}:
\begin{itemize}
	\item $R_i := c$
	\item $R_i := R_j$
	\item $R_i := [R_j]$
	\item $[R_i] := R_j$
	\item $R_i := R_j$ \texttt{op} $R_k$ - aritmetické instrukce, \texttt{op} $\in \{+, -, *, /\}$ \\ nebo $R_i := R_j$ \texttt{op} $c$
	\item \texttt{if} $(R_i$ \texttt{rel} $R_j)$ \texttt{goto} $l$ - podmíněný skok, \texttt{rel} $\in \{=, \neq, \leq, \geq, <, >\}$ \\ 
				nebo \texttt{if} $(R_i$ \texttt{rel} $c)$ \texttt{goto} $l$
	\item \texttt{goto} $l$ - nepodmíněný skok
	\item $R_i :=$ \texttt{READ}$()$ - čtení ze vstupu
	\item \texttt{WRITE}$(R_i)$ - zápis na výstup
	\item \texttt{halt} - zastavení programu
\end{itemize}
Všechny uvedené instrukce jsou v aplikaci plně podporovány a jejich detailní popis se nachází v kapitole \ref{sec:machine_impl}.

\section{Simulace Turingova stroje strojem RAM}
\label{sec:simulationTheory}
Jelikož Turingův stroj pracuje se znaky, se kterými stroj RAM pracovat neumí, je zapotřebí nějakého slovníku, 
co by nám ke každému znaku přiřadil číselnou hodnotu. Například tímto předpisem:
\[
  \Sigma \;\longrightarrow\; \mathbb{N},\qquad 
  a_1 \rightarrow 1,\; a_2 \rightarrow 2,\;\dots,\; a_n \rightarrow n
\]
Hodnota 0 je rezervována pro znak $\square$, aby z počátku nulová (nezapsaná) buňka paměti RAM odpovídala prázdnému symbolu na pásce.
Samotný program stroje RAM lze tvořit následujícím způsobem:
\begin{enumerate}
	\item Program začne skokem na aktuální stav.
	\item Pro každý stav definuj návěští, které si načte aktuální symbol na pásce. 
		Poté následuje série skoků, které \enquote{rozřadí} běh programu do konkrétních stavů s daným symbolem.
	\item Pro každý stav s konkrétním symbolem definuj návěští. Aktualizuj symbol na pozici čtecí hlavy podle přepisovacího pravidla přechodové funkce a vykonej posun ($\{-1, 0, +1\}$). 
		Pokud je nový stav koncový - konec programu (\texttt{halt}), jinak skoč na tento nový stav (zpět krok 2).
\end{enumerate}