\chapter{Úvod}
\label{sec:Introduction}
Cílem této semestrální práce je vytvoření aplikace pro simulaci Turingova stroje strojem RAM, sloužící k výuce teoretické informatiky. 
Aplikace obsahuje simulaci Turingova stroje s předpřipravenými příklady, které jsou následně přeloženy do kódu stroje RAM. 
Oba stroje lze poté zároveň krokovat a v reálném čase vidět obsah pásek a paměti, včetně stavu a pozice strojů. 
K předpřipraveným přikladům lze definovat i své vlastní stroje, které je možné sdílet mezi jednotlivými zařízeními.

Práce je rozdělena do několika kapitol, kdy ve \ref{sec:theory}. kapitole je popsán teoretický základ obou strojů nutný pro pochopení průběhu simulace, 
následován popisem několika vybraných předpřipravených strojů a popisem samotného algoritmu simulace Turingova stroje. 
V další kapitole je probrán stručný popis použitých technologií, jak jsou použity a jaké výhody do aplikace přináší.
V neposlední řadě, v kapitole \ref{sec:implementation}, je probrána samotná webová aplikace, její návrh, implementace, uživatelské rozhraní a jsou zde také natíněny možné způsoby rožšíření aplikace.

Webová aplikace je v průběhu obhajoby dostupná na adrese \texttt{https://ram.koberskyj.cz/}, zprostředkované přes Github Pages \cite{githubpages}.
\endinput